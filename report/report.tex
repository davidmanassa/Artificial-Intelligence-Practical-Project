\documentclass{article}

\usepackage[utf8]{inputenc} % Allow more characters

\usepackage{graphicx} % images

\usepackage{xcolor}

\usepackage{listings} % code quotes
\lstset{
  basicstyle=\ttfamily,
  columns=fullflexible,
  frame=single,
  breaklines=true,
  postbreak=\mbox{\textcolor{red}{$\hookrightarrow$}\space},
}

\title{Relatório de Trabalho Prático}
\author{David Pires e Pedro Maria}
\date{Janeiro 2020}

\begin{document}

\begin{titlepage}
    \maketitle

    \includegraphics[scale=0.5]{logo_ubi_vprincipal.jpg}

\end{titlepage}

\section{Conteúdo}

\tableofcontents

\newpage

\section{Motivação}

Somos fortes

\section{Objetivos}

Chegar ao fim

\newpage

\section{Implementação}

\subsection{Variáveis auxiliares}

\paragraph{Para a conclusão deste trabalho, recorremos à criação de várias váriàveis auxiliares, para o manuseamento das informações obtidas pelo robô e pedidas pelo utilizador do programa.}

\subsubsection{Lista das salas}
\begin{lstlisting}[language=Python]
room_list = [(1,-15.6,-3.0,-3.6,-0.8), (2,-12.0,-1.4,-8.9,4.8), (3,-10.6,4.8,3.6,7.9), (4,-4.6,-0.8,-0.8,4.8), (5,-15.6,-1.4,-12.5,3.0), (6,-15.6,3.0,-12.5,7.9), (7,-15.6,7.9,-10.6,11.1), (8,-10.6,7.9,-5.6,11.1), (9,-5.6,7.9,-0.5,11.1), (10,-0.5,7.9,3.6,11.1), (11,-0.8,1.4,3.6,4.8), (12,-0.8,-0.8,3.6,1.4), (13,-8.9,-0.8,-6.5,4.8),(14,-6.5,-0.8,-4.6,4.8)]
\end{lstlisting}

Esta é uma lista pré definida no programa que contem as coordenadas de cada sala. Esta lista é composta por tupulos do tipo (roomNumber, x1, y1, x2, y2) em que roomNumber corresponde ao numero da sala, x1 e y1, correspondem às coordenadas do canto inferior esquerdo da sala e x2, y2, correponde às coordenadas do canto superior direito da sala.

\subsubsection{Lista de objetos}
\begin{lstlisting}[language=Python]
object_list = []
\end{lstlisting}  

Esta é uma lista, que é incrementada dinamicamente à medida que o robô descobre um novo objeto. Esta lista será composta por tupulos, com a seguinte estrutura: (roomNumber, objectName, objectID), em que roomNumber corresponde ao numero da sala onde foi encontrado o objeto, objectName corresponde ao nome do objecto e objectID corresponde ao seu ID.

\subsubsection{Lista de pontos de interesse}
\begin{lstlisting}[language=Python]
point_list = []
\end{lstlisting}

Esta é uma lista incrementada dinamicamente com pontos de interesse, que neste caso se podem traduzir em "portas" (menos nos casos se passagem de um corredor para outro).
Esta lista é composta por tuplos, com a seguinte estrutura: (uuid, x, y, roomA, roomB), onde:
\newline - uuid é um ID único aleatoriamente gerado, representante deste ponto. 
\newline - x e y são as coordenadas do ponto.
\newline - roomA e roomB são as salas entre o ponto (visto que este é um ponto correspondente à passagem de uma sala para outra).

\subsubsection{Garfo dos pontos de interesse}
\begin{lstlisting}[language=Python]
  from collections import defaultdict

  class Graph():
      def __init__(self):
          self.edges = defaultdict(list)
          self.weights = {}
      
      def add_edge(self, from_node, to_node, weight):
          self.edges[from_node].append(to_node)
          self.edges[to_node].append(from_node)
          self.weights[(from_node, to_node)] = weight
          self.weights[(to_node, from_node)] = weight
  
      def getWeight(self, from_node, to_node):
          return self.weights[(from_node, to_node)]  
\end{lstlisting}
\begin{lstlisting}[language=Python]
graph = Graph()
\end{lstlisting}

Para isto trabalhamos numa classe feita originalmente por Ben Keen (Link na Bibliografia). Este é um grafo bi-dirigido com pesos, em que cada peso corresponde à distância de um ponto ao outro e as ligações corresponderão a pontos possíveis de alcançar sem paredes como obstáculos. Cada nodo é um uuid, representante de um ponto na lista de pontos.

\includegraphics[scale=0.4]{rosexample.png}

\newpage
\subsection{Funções auxiliares}  
  
\paragraph{Neste trabalho recorremos à criação de várias funções auxiliares, de forma a facilitar o nosso trabalho. Aqui vamos explicar algumas dessas funções.}

\subsubsection{Calcular a distância entre dois pontos (x1, y1) e (x2, y2)}
\begin{lstlisting}[language=Python]
    import math

    def calculateDistance(x1, y1, x2, y2):
        dx = x2-x1
        dy = y2-y1
        distance = math.sqrt((dx**2) + (dy**2))
        return distance
\end{lstlisting}

Esta função localiza-se no ficheiro CoordHelper.py (Decidimos criar este ficheiro separado, para caso, venha a ser necessário criar mais funções auxiliares relacionadas com Coordenas) e retorna um valor que representa a distância entre dois pontos no plano x, y.

\subsubsection{Registo de um novo ponto de interesse}
\begin{lstlisting}[language=Python]
  def new_point(coordX, coordY, roomA, roomB):
	uid = uuid.uuid4()
	point_list.append((uid, coordX, coordY, roomA, roomB))
	for point in point_list:
		if point[3] == roomA or point[3] == roomB or point[4] == roomA or point[4] == roomB:
			distance = CoordHelper.calculateDistance(point[1], point[2], coordX, coordY)
			graph.add_edge(uid, point[0], distance)
\end{lstlisting}

Esta função é chamada sempre que o robô passa de uma sala para outra. Esta usa a biblioteca uuid para gerar um novo uuid, e guarda os dados na lista de pontos e para cada ponto de interesse alcançável sem obstáculos, calcula a distância e adiciona uma ponte no grafo.

\subsubsection{Algoritmo dijsktra}
\begin{lstlisting}[language=Python]
  def dijsktra(graph, initial, end):
  # shortest paths is a dict of nodes
  # whose value is a tuple of (previous node, weight)
  shortest_paths = {initial: (None, 0)}
  current_node = initial
  visited = set()
  
  while current_node != end:
      visited.add(current_node)
      destinations = graph.edges[current_node]
      weight_to_current_node = shortest_paths[current_node][1]

      for next_node in destinations:
          weight = graph.weights[(current_node, next_node)] + weight_to_current_node
          if next_node not in shortest_paths:
              shortest_paths[next_node] = (current_node, weight)
          else:
              current_shortest_weight = shortest_paths[next_node][1]
              if current_shortest_weight > weight:
                  shortest_paths[next_node] = (current_node, weight)
      
      next_destinations = {node: shortest_paths[node] for node in shortest_paths if node not in visited}
      if not next_destinations:
          return "Route Not Possible"
      # next node is the destination with the lowest weight
      current_node = min(next_destinations, key=lambda k: next_destinations[k][1])
  
  # Work back through destinations in shortest path
  path = []
  while current_node is not None:
      path.append(current_node)
      next_node = shortest_paths[current_node][0]
      current_node = next_node
  # Reverse path
  path = path[::-1]
  return path
\end{lstlisting}

Para isto usamos o código feito por Ben Keen (Link na Bibliografia). Esta função retornará uma lista de pontos, que representará o menor caminho a percorrer para ir de um ponto até outro.

\newpage
\subsection{Resposta às perguntas}

\subsubsection{Pergunta 1: How many rooms are not occupied?}
\begin{lstlisting}[language=Python]
  def question1():
	counterOccuped = 0
	counterNotKnown = 0
	for roomNumber in range(5, len(room_list) + 1):
		counterObj = 0
		for obj in object_list:
			if obj[0] == roomNumber:
				counterObj += 1
				if obj[1] == "person":
					counterOccuped += 1
		if counterObj == 0:
			counterNotKnown += 1
	print( " There are %d rooms not occuped by people in %d known rooms. " % (((10 - counterNotKnown) - counterOccuped), (10 - counterNotKnown)) ) 
\end{lstlisting}

\subsubsection{Pergunta 2: How many suites did you find until now?}
\begin{lstlisting}[language=Python]
  def question2():
	counter = 0
	for roomNumber in range(1, len(room_list) + 1):
		if (getRoomType(roomNumber) == "Suite room"):
			counter += 1
	print( " I've found %d Suite rooms so far. " % counter )
\end{lstlisting}

\subsubsection{Pergunta 3: Is it more probable to find people in the corridors or inside the rooms?}
\begin{lstlisting}[language=Python]
  def question3():
	counterHall = 0
	counterRooms = 0
	for obj in object_list:
		if obj[1] == "person":
			if obj[0] <= 4:
				counterHall += 1
			else:
				counterRooms += 1
	if counterHall > counterRooms:
		print( " Is more likely to meet people in the halls. " )
	elif counterHall < counterRooms:
		print( " Is more likely to meet people in the rooms. " )
	elif counterHall == 0 and counterRooms == 0:
		print ( "I don't know any person yet. " )
	else:
		print( " The probability of find people in rooms or in the halls is equal. " )
\end{lstlisting}

\subsubsection{Pergunta 4: If you want to find a computer, to which type of room do you go to?}
\begin{lstlisting}[language=Python]
  def question4():
	roomNumber = -1
	for obj in object_list:
		if obj[1] == "computer":
			if getRoomType(obj[0]) == "Meeting room" or getRoomType(obj[0]) == "Generic room": # Only for privacy :)  
				roomNumber = obj[0]
				break
			roomNumber = obj[0]
	if roomNumber == -1:
		print( " I don't know any room with a computer. " )
	else:
		print( " Go to room number %d to find a Computer. " % roomNumber )
\end{lstlisting}

\subsubsection{Pergunta 5: What is the number of the closest single room?}
\begin{lstlisting}[language=Python]
  def closestSingleRoom(atualX, atualY):
	min_room = -1
	min_distance = 9999999
	for room in room_list:
		if (getRoomType(room[0]) == "Single room"):
			tempDistance = calculateDistance(atualX, atualY, dijsktraRooms(match_room(atualX, atualY), room[0]))
			if (tempDistance < min_distance):
				min_distance = tempDistance
				min_room = room[0]
	return min_room

def question5():
	csr = closestSingleRoom(x_ant, y_ant)
	if csr != -1:
		print("The closest Single room is %d." % csr)
	else:
		print("I don't know any Single room yet.")
\end{lstlisting}

Para responder a esta pergunta recorremos ao cálculo da distância de onde o robô está até cada Single room conhecida, calculando o menor caminho para cada Single room com o algoritmo dijsktra e comparando as distâncias.
Assim retornará a Single room mais perto (com menor distância, a andar).

\subsubsection{Pergunta 6: How can you go from the current room to the elevator?}
\begin{lstlisting}[language=Python]
  def question6():
	roomPath = getRoomPath(dijsktraRooms(match_room(x_ant, y_ant), -1), match_room(x_ant, y_ant))
	roomPath = roomPath[1:-1]
	result = " Visit the follow rooms to go to the Elevator: "
	for room in roomPath:
		result += str(room) + "  "
	print(result)
\end{lstlisting}

Para responder a esta pergunta usamos o algoritmo dijsktra para caulcular o menor caminho desde a sala atual do robô até ao elevador (representado como a sala -1). Retornamos uma lista ordenada das salas a visitar até chegar ao elevador.

\subsubsection{Pergunta 7: How many books do you estimate to find in the next 2 minutes?}
\begin{lstlisting}[language=Python]
  startTime = time.time()
\end{lstlisting}
\begin{lstlisting}[language=Python]
  def question7():
	actualTime = time.time()
	counterBooks = 0
	for obj in object_list:
		if obj[1] == "book":
			counterBooks += 1
	result = (120 * counterBooks) / (actualTime - startTime)
	print( " I think I will find %d books in the next 2 minutes. " % int(result))
\end{lstlisting}

Para responder a esta pergunta recorremos à biblioteca time, guardando no inicio do programa o tempo o tempo de inicio, e cada vez que for feita esta questão um tempo. Para responder à pergunta será dada uma resposta com base no seguinte cálculo:
\newline (120 * numberOfBooks) / timeInterval
\newline Em que, numberOfBooks corresponde ao numero de livros encontrados pelo robô até agora e timeInterval corresponde ao tempo decorrido em segundos desde o inicio do programa.

\subsubsection{Pergunta 8: What is the probability of finding a table in a room without books but that
has at least one chair?}
\begin{lstlisting}[language=Python]
  def question8():
	counterRoomWithChairAndNotBook = 0
	counterRoomWithTableAndChairAndNotBook = 0
	for room in range(5, 14):
		counterBook = 0
		counterChair = 0
		counterTable = 0
		for obj in object_list:
			if obj[0] == room:
				if obj[1] == "chair":
					counterChair += 1
				if obj[1] == "book":
					counterBook += 1
				if obj[1] == "table":
					counterTable += 1
		if counterChair > 0 and counterBook == 0:
			counterRoomWithChairAndNotBook += 1
		if counterChair > 0 and counterTable > 0 and counterBook == 0:
			counterRoomWithTableAndChairAndNotBook += 1
	if counterRoomWithChairAndNotBook == 0:
		print( " I don't know any room without books but that has at least one chair yet. " )
	else:
		result = counterRoomWithTableAndChairAndNotBook / counterRoomWithChairAndNotBook
		print( " The probability of finding a table in a room without books but that has at least one chair is %d. " % result )
\end{lstlisting}

Para responder a esta pergunta usamos a Regra de Bayes, calculando \[P(T | C, ~B)\] em que:
\newline - T corresponde à probabibilidade de encontrar uma o mais Mesas em uma sala.
\newline - C corresponde à probabibilidade de encontrar uma o mais Cadeiras em uma sala.
\newline - B corresponde à probabibilidade de encontrar um o mais Livros em uma sala.
\newline \[P(T | C, ~B)\] Coresponde à probabibilidade de encontrar uma ou mais mesas em uma sala sabendo que contem uma ou mais cadeiras e não contem livros. 

\newpage
\section{Conclusão}

\paragraph{Neste trabalho concluimos que bla bla }

\subsection{Partição do trabalho}

Perguntas 1 a 4: Pedro Maria
\newline Perguntas 5 a 8: David Pires
\newline Relatório: Trabalho conjunto
\newline Apresentação: Trabalho conjunto

\section{Bibliografia}

- http://benalexkeen.com/implementing-djikstras-shortest-path-algorithm-with-python/

\end{document}